% Manual for the cpw2000 program
%
% To generate the printed version:
%
% pdflatex manual
%
%
\documentclass[11pt]{article}

\tolerance 10000
\textheight 24cm
\textwidth 16cm
\oddsidemargin 1mm
\topmargin -20mm

\parindent=0cm
\baselineskip=12pt
\parskip 5pt

\usepackage{amsmath}
\usepackage{graphicx}


% \NewDocumentEnvironment{fdfentry}{ }
% {   %
%     % Create correct box-setups
%     \bgroup
%     \list{%
%     }{ %
%         % Define list
%         \topsep=4pt
%         \partopsep=0pt
%         \leftmargin=2em\itemindent-\leftmargin %
%         \listparindent=0pt
%         \itemsep=4pt
%         %\def\makelabel##1{\hss##1} %
%     }%
%     % Create entry line of the argument
%     \fdfentryline %
% }
% {%
%     \endlist%
%     \egroup
% }



\begin{document}

% TITLE PAGE
% --------------------------------------------------------------

\begin{titlepage}

\begin{center}
~
\vfill
\vspace{1cm}
{\Huge {\bf cpw2000 User Guide}}
\par\vspace{3cm}
\hrulefill
\par\vspace{3cm}
{\Large {\bf Version 5.12, Draft \today}}
\par\vspace{2cm}
\hrulefill

{\Large Jos\'e Lu\'{\i}s Martins \\
Departamento de F\'{\i}sica, Instituto Superior T\'ecnico, Lisboa, Portugal (Retired)

INESC MN, Lisboa, Portugal

\texttt{jlmartins@inesc-mn.pt}

\texttt{jose.l.martins@tecnico.ulisboa.pt}
}
\vfill
\end{center}

\end{titlepage}
% END TITLE PAGE
% --------------------------------------------------------------

\tableofcontents
\newpage

\section{Introduction}
\label{sec:intro}

This is the current version of a old code that calcultes the electronic structure of crystals
within the pseudopotential approximation with a plane-wave basis.

The very original code was written by Sverre Froyen at the University of California Berkeley.
I do not know what was the first time his version of the code, as there was already an earlier
pseudopotential code at Berkeley, I would guess that
Physical Review B, 26, 3258 (1983) https://journals.aps.org/prb/abstract/10.1103/PhysRevB.28.3258
already used that version.

After a long conversation with Roberto Car about the seminal Car-Parrinello work, I wrote at Berkeley
in Marvin Cohen's group the
code for the iterative diagonalization of the plane-wave matrices
Physical Review B 37, 6134 (1988).

Since then many people contributed to the code.   At Minnesota Norm Troullier
and the group of Jim Chelikowsky contributed to the development.
Renata Wentzcovitch implemented the original molecular dynamics and the variational
cell shape molecular dynamis.  Nadia Binggeli implemented the Langevin molecular dynamics.
At Lisbon the largest contribution was from Carlos Loia Reis.

\section{Installation}
\label{sec:install}

\subsection{Step 1: Downloading and extracting the archive}
\label{sec:step1}

The code is available from GitHub

\noindent\texttt{https://github.com/jlm785/cpw2000}

If you downloaded the \texttt{cpw2000-5.x.y.tar.gz} file (where \texttt{x} and \texttt{y}
are the minor version numbers) just extract it,

\noindent\texttt{\$ tar xzf cpw2000-5.x.y.tar.gz}

and you will have a \texttt{cpw2000-5.x.y} directory

If you cloned the git you already have the relevant code in the main directory.

\subsection{Step 2: Generating the documentation}
\label{sec:step2}

The documentation is in the \texttt{cpw2000-5.x.y/Doc} directory.  It include this file, and the means to generate a detailed
description of the code for developers.

For that detailed description you need doxygen (\texttt{https://www.doxygen.nl}) and
graphviz (\texttt{https://graphviz.org/}), both are
available on most distributions.  To check that your computer has doxygen and graphviz installed
run the commands

\noindent\texttt{cpw2000-5.x.y/Doc\$ dpkg -s doxygen}

\noindent\texttt{cpw2000-5.x.y/Doc\$ dpkg -s graphviz}

on a Debian based distribution (Ubuntu et al), or use equivalent tools (yum, rpm, dnf zypper,...) or the relevant GUI.

If they are not available just install them

\noindent\texttt{cpw2000-5.x.y/Doc\$ sudo apt install doxygen}

\noindent\texttt{cpw2000-5.x.y/Doc\$ sudo apt install graphviz}

again for Debian based distros, or use equivalent tools (yum, rpm, dnf zypper,...) or the relevant GUI.

Finally run doxyfile in the doxy directory

\noindent\texttt{cpw2000-5.x.y/Doc/doxy\$ doxygen Doxyfile}

If you know what you are doing you can edit \texttt{cpw2000-5.x.y/Doc/doxy/Doxyfile}.

Finally it is useful to create a link to the file \texttt{cpw2000-5.x.y/Doc/html/index.html}

\noindent\texttt{cpw2000-5.x.y/Doc\$ ln -s html/index.html documentation.html}

Opening that link in your browser will allow you to see the documentation for (almost) every file in the code.

\subsection{Step 3: Compiling the code}
\label{sec:step3}

The code has been tested with several Fortran compilers, \texttt{ifort} (from Intel oneAPI),
\texttt{gfortran} (from gnu),
\texttt{pgfortran} (from Portland), and even LLVM (experimental Intel compiler).

The choice of compiler is in the \texttt{cpw2000-5.x.y/Src/make.inc} file.  You may edit it for your convenience,
but in principle you have to make just two choices.  The first is to identify which CPU you are using,
as it narrows the choice of possible compilers.  This is done by commenting/uncommenting
the lines below \texttt{\#   compiler for the job}.

The second is the compiler you want, and is the key decision.  Again comment/uncomment the lines below
\texttt{\#   Suggestions for compilers}.


For any other modification of the \texttt{make.inc} file it is assumed you know what you are doing...
The \texttt{make.inc} file was adapted from the {\sc Wannier90} code
\texttt{https://wannier.org/}.
As compilers may evolve with time, if you run into problems check
the documentation and corresponding texttt{make.inc} file from that distribution.
Another code that uses a \texttt{make.inc} file is {\sc elk}
\texttt{https://elk.sourceforge.io/}.  There you can find suggestions for
compiler options.

To compile just go to \texttt{cpw2000-5.x.y/Src} and type make

\noindent\texttt{cpw2000-5.x.y\$ cd Src}

\noindent\texttt{cpw2000-5.x.y/Src\$ make}

You should get the main library \texttt{libcpw{\_}}\textit{compiler}\texttt{.a}
and a few executables \texttt{cpw{\_}}\textit{compiler}\texttt{.exe}, etc...
where \textit{compiler} is the name of the chosen compiler.
As you probably are not interested in comparing compilers,
it is advisable to create links without the compiler name.
For example, after compiling with \texttt{ifort}, I usually create the links

\noindent\texttt{cpw2000-5.x.y\$ ln -s Src/cpw\_ifort.exe cpw.exe}

\noindent\texttt{cpw2000-5.x.y\$ ln -s Src/cpw\_post\_process\_ifort.exe cpw\_post\_process.exe}

\noindent\texttt{cpw2000-5.x.y\$ ln -s Src/libpw\_ifort.a libpw.a}

In the examples it will be assumed that those links are created, but obviously
the links could be anywhere (in a \texttt{bin} folder for example).




\section{First time run of the code}
\label{sec:run}

I assume you have a working directory separate from the source.  In the following I will call it
\texttt{cpw2000-5.x.y/WORK}


\subsubsection{Files required to run the code}


You will need a file with the crystal description and a pseudopotential file for each type of atom
in the crystal.

The file with the crystal description is called \texttt{cpw.in}.  You must have such a file
in your working directory.  You can find such a file for almost all elements in
\texttt{cpw2000-5.x.y/Structures/Elements}
and files for other structures in the other subdirectories of texttt{cpw2000-5.x.y/Structures}.
The format of the crystal description is the same as in the {\sc siesta} code,
texttt{https://siesta-project.org/siesta/index.html} so you may find descriptions
for other crystals in the ``net''.  {\bf Beware} that presently the code only accepts
{\bf fractional} lattice coordinates.

For back-compatibility, a file \texttt{PW.DAT} can be used to read the crystal structure,
k-point sampling and energy cutoff.  But other parameters cannot be set by that file.

There is also a tool, \texttt{Tools/gen{\_}PW.f90},  to obtain a \texttt{cpw.in} file by answering a few questions.


You will have to generate the pseudopotentials for each atom, or find someone who has done that
for you and have a file in the relevant format.  The name of the file is \texttt{Xy{\_}POTKB{\_}F.DAT} where
\texttt{Xy} is the one or two character chemical symbol.  This file should be (or at least a link) on
your working directory, or an explicit path should be specified in \texttt{cpw.in}.

The good news is that the code to generate pseudopotentials is available from GitHub

\noindent\texttt{https://github.com/jlm785/pseudopotential}

and if you run the test on \texttt{Validation} after you followed the relevant instructions
to install the code you will get pseudopotential files for all elements.

The bad news is that some of those pseudopotentials where not extensively tested, and even those
that are reported as tested may not be what you want.  Just test them in a simple case
before you proceed.  As pseudopotentials calculations should reproduce first principles
calculations in \texttt{cpw2000-5.x.y/Structures/Elements} you can find
\texttt{elk.in} files that you can use to run LAPW all-electron calculations with the {\sc elk} code,
\texttt{https://elk.sourceforge.io/}.  Testing the pseudopotential will spare you a lot of grief
later.

If you want just a ``quick and dirty'' calculation than you can use the pseudopotentials
from \texttt{Validation} of the pseudopotential code.


\subsection{Running the code}
\label{sec:run-code}

Once you have the required files you run the code with

\noindent\texttt{cpw2000-5.x.y/WORK\$ ../cpw.exe}

if you created the appropriate links, or

\noindent\texttt{cpw2000-5.x.y/WORK\$ ../Src/cpw{\_}}\textit{compiler}\texttt{.exe}

otherwise.   The code will write to default output, so redirect it with

\noindent\texttt{cpw2000-5.x.y/WORK\$ ../cpw.exe > output.dat}

or

\noindent\texttt{cpw2000-5.x.y/WORK\$ ../cpw.exe | tee output.dat}.

You will get a \texttt{PW{\_}RHO{\_}V.DAT} file with the self-consistent potential.  By running

\noindent\texttt{cpw2000-5.x.y/WORK\$ ../cpw{\_}post{\_}process.exe}

you will start an interactive analysis of the results.

If both executables ran without errors you are set.

\section{The \texttt{cpw.in} file}

The \texttt{cpw.in} syntax is similar to the input file from the SIESTA code.
The parser is different, \texttt{esdf} instead of \texttt{fdf}.  The keywords
are even the same when possible.  Therefore part of the text of this section
was pilfered from the SIESTA documentation.

The \texttt{cpw.in}
contains all the physical data of the system and the parameters of
the simulation to be performed.

This file is written in a special format called {\sc esdf}, developed by
Chris J. Pickard. This format allows data to be
given in any order, or to be omitted in favor of default values.
It is used in other electronic structure codes such as {\sc casino} and
{\sc parsec}.   However to be consistent with {\sc siesta} (which uses the fdf format) the special character
indicating a block structure has been changed.

Here we offer a glimpse of the rules:

\begin{itemize}

\item[$\bullet$] The syntax is a 'data label' followed by its value.
Values that are not specified in the datafile are assigned
a default value.

\item[$\bullet$] The labels are case insensitive, and characters - \_ .
in a data label are ignored. Thus, LatticeConstant and
lattice{\_}constant represent the same label.

\item[$\bullet$] All text following the \# character is taken as comment.

\item[$\bullet$] Logical values can be specified as T, true, .true.,
yes, F, false, .false., no.

\item[$\bullet$] Character strings should \textbf{not} be in apostrophes.

\item[$\bullet$] Real values which represent a physical magnitude must be
followed by its units.
It is important to include a decimal point in a real number to distinguish
it from an integer, in order to prevent ambiguities when mixing the types
on the same input line.

\item[$\bullet$] Complex data structures are called blocks and are
placed between `\%block label' and a `\%endblock label'
(without the quotes).

\item[$\bullet$] If the same label is specified twice, the first one takes precedence.

\item[$\bullet$] If a label is misspelled it will not be recognized (there is no
  internal list of ``accepted'' tags in the program). You can check
  the actual value used by {\sc cpw2000} by looking for the label in the
  output (by default standard output).

\end{itemize}

\noindent
This is an example for silicon:

\begin{verbatim}
          LatticeConstant             10.2629     bohr

          %block LatticeVectors
                   0.00000000         0.50000000         0.50000000
                   0.50000000         0.00000000         0.50000000
                   0.50000000         0.50000000         0.00000000
          %endblock LatticeVectors

          NumberOfSpecies                1

          NumberOfAtoms                2

          %block Chemical_Species_Label
                         1       14   Si
          %endblock Chemical_Species_Label

          AtomicCoordinatesFormat     Fractional

          %block AtomicCoordinatesAndAtomicSpecies
                   0.12500000      0.12500000      0.12500000      1     #  Si       1
                  -0.12500000     -0.12500000     -0.12500000      1     #  Si       1
          %endblock AtomicCoordinatesAndAtomicSpecies

          StructureSource               Experiment

          #------------------------------------------------
          # Energy cutoff, bands,  and Brillouin mesh
          #------------------------------------------------

          PWEnergyCutoff                  12.0000      hartree

          NumberOfEigenStates                  10

          %block kgrid_Monkhorst_Pack
                         4       0       0       0.500000
                         0       4       0       0.500000
                         0       0       4       0.500000
          %endblock kgrid_Monkhorst_Pack
\end{verbatim}




\subsection{General stuff}

\begin{itemize}

%\begin{fdfentry}{SystemLabel}[string]<siesta>
\item{\bf SystemLabel}

  A \emph{single} word (max. 20 characters \emph{without blanks})
  containing a nickname of the system. Reserved to be used to name output files in the future.

%\end{fdfentry}

\item{\bf PrintingLevel} (integer)

   Defines the detail of the printout.  May take the values 1,2, or 3.

  \textit{Default value:}  1 (for molecular dynamics) or 3 (single geometry).

   The higher the value the more details will be printed.
   Remember that too much information is noise.  1 is recommended for molecular
   dynamics.  3 for single calculation, or when things seem to go wrong.

\item{\bf MD.TypeOfRun} (string)

   Choice on how the atoms move.  Molecular dynamics or geometry optimization.

  \textit{Default value:}  ONE

  With the default it will run a single geometry.
  See below for the other options to run molecular dynamics calculations.

  The value in \texttt{cpw.in} may be overridden by the first argument
  of the executable.

  \noindent\texttt{cpw2000-5.x.y/WORK\$ ../cpw.exe MICRO > output.dat}

  will perform a microcanonic molecular dynamics calculation irrespective
  of the value in \texttt{cpw.in}.

\end{itemize}


\subsection{Crystal descriptors}

These are the lines that describe the crytal structure.

\begin{itemize}

%\begin{fdfentry}{LatticeConstant}[length]
\item{\bf LatticeConstant} (length)

  A physical value, requiring a real number followed by the units.
  Accepted units are: \texttt{bohr, ang, nm, m}.
  This is just to define the scale of the lattice vectors.

  \textit{Default value:} 1.0 bohr.

  The code stops if the units are not present or not recognized.
  Internally the code uses atomic units (bohr).

%\begin{fdfentry}{LatticeVectors}[block]
\item{\bf LatticeVectors} (block)

  The cell vectors are read in units of the lattice constant defined
  above.  They are read as a matrix, each
  vector being one line.

  \textit{Default value:}  Unit matrix.

  The internal representation of the lattice vectors in the code is
  by their metric tensor.  So the original orientation is space is lost!
  When the code needs to read or print information with
  orientation content it will use some {\it canonic} vectors
  based on the lattice symmetry.   Those are printed
  at the start of the SCF code, and before questions to the user
  in the post-processing code.  Just pay attention,
  they may be {\bf different} from what is written in \texttt{cpw.in}!
  See the section on output for more details.

  The code checks if the entered value is near a rational number or
  the square root of a rational number (with low denominators).
  If it is the case it will use that modified value and print a warning.
  This allows to have an exact symmetry, and the modifications are always
  small.


%\begin{fdfentry}{NumberOfSpecies}[integer]<\nonvalue{lines in \fdf{ChemicalSpeciesLabel}}>
\item{\bf NumberOfSpecies} (integer)

  Number of different atomic species in the simulation.  It must be
  the number of lines in the {\bf ChemicalSpeciesLabel} block.

  \textit{Default value:}  1

  If not present or nin positive it will use the number of lines in
  {\bf ChemicalSpeciesLabel}.

  In case of inconsistency the execution terminates.

%\begin{fdfentry}{NumberOfAtoms}[integer]<\nonvalue{lines in \fdf{AtomicCoordinatesAndAtomicSpecies}}>
\item{\bf NumberOfAtoms} (integer)

  Number of atoms in the simulation.  It is the number of lines in the block
  {\bf AtomicCoordinatesAndAtomicSpecies}.

  \textit{Default value:}  1

  In case of inconsistency the execution terminates.


%\begin{fdfentry}{ChemicalSpeciesLabel}[block]
\item{\bf ChemicalSpeciesLabel} (block)

  It specifies the different chemical species that are
  present, assigning them a number for further identification.
  cpw2000 recognizes the different atoms by the given atomic number.

  One line for each species.
  The first number in a line is the species number, it is followed by
  the atomic number, and then by the chemical symbol.  From H to Og
  all chemical symbols are recognized.  There is an extra chemical
  symbol, ZZ with a number of protons (atomic number) that can be zero
  to allow adding special pseudopotentials to the crystal.

  In case of inconsistency the execution terminates.


%\begin{fdfentry}{AtomicCoordinatesFormat}[string]<Bohr>
\item{\bf AtomicCoordinatesFormat} (string)

  Character string to specify the format of the atomic positions in
  input.  {\bf It is not used!!!}  It is here for future use and
  compatibility with {\sc siesta}.

%These can be expressed in four forms:

%   \begin{fdfoptions}
%     \option[Bohr|NotScaledCartesianBohr]%
%     \fdfindex*{AtomicCoordinatesFormat:Bohr}%
%     \fdfindex*{AtomicCoordinatesFormat:NotScaledCartesianBohr}%
%
%     atomic positions are given directly in Bohr, in Cartesian
%     coordinates
%
%     \option[Ang|NotScaledCartesianAng]%
%     \fdfindex*{AtomicCoordinatesFormat:Ang}%
%     \fdfindex*{AtomicCoordinatesFormat:NotScaledCartesianAng}%
%
%     atomic positions are given directly in \AA ngstr\"om, in Cartesian
%     coordinates
%
%     \option[ScaledCartesian]%
%     \fdfindex*{AtomicCoordinatesFormat:ScaledCartesian}%
%
%     atomic positions are given in Cartesian coordinates, in units of
%     the lattice constant
%     \begin{itemize}
%
%        \item{Fractional}
%
%        Atomic positions are given referred to the lattice vectors
%     \end{itemize}

%\begin{fdfentry}{AtomicCoordinatesAndAtomicSpecies}[block]
\item{\bf AtomicCoordinatesAndAtomicSpecies} (block)

  Block specifying the position and species of each atom.  One line
  per atom, with three reals followed by one integer.
  In total the number of lines indicated by {\bf NumberOfAtoms}
  must be present.

  The three reals indicate the atomic positions in
  {\bf FRACTIONAL lattice coordinates} followed by the
  species of atom on that position,
  as identified in the {\bf Chemical\_Species\_Label} block.

  In case of inconsistency the execution terminates.

\item{\bf StructureSource} (string)

  Information on the source of the crytal structure.
  Not used by the code,


\end{itemize}



\subsection{Major Self Consistent Field parameters}

This are the parameters whose values you would mention in a paper to allow reproducibility.

\begin{itemize}

\item{\bf PWEnergyCutoff} (energy)

  Energy cutoff of the plane wave basis set expansion.  Real value followed
  by the energy unit.
  Accepted units are: \texttt{eV, Ry, Hartree}, and a few other.

  \textit{Default value:}  5 Hartree

  Internally the code uses Hartrees.


% \begin{fdfentry}{NumberOfEigenStates}[integer]<\nonvalue{all orbitals}>
%  \fdfdepend{Diag!Algorithm}
\item{\bf NumberOfEigenStates} (integer)

  This parameter indicates the number of eigenstates
  to be calculated.

  \textit{Default value:}  10

  In some future the default should be slightly larger than half the number of electrons.

  By choosing a low value the cost of the diagonalization
  may be reduced by finding fewer eigenstates.
  However choosing a slightly larger number of active eigenstates
  than the bare minimum may help converge faster the occupied
  eigenstates and therefore the overall calculation.
  Note, that if the electronic temperature is greater than zero then
  the number of partially occupied states increases, depending on the band gap.
  The value specified must be greater than the number of occupied states
  (at least the number of electrons divided by two for
  a non-spin-polarized calculation) and
  less than the number of basis functions (which is extremly large for plane waves).

%\begin{fdfentry}{kgrid!MonkhorstPack}[block]<$\Gamma$-point>
\item{\bf kgridMonkhorstPack} (block)

  Specifies the Fourier integration grid, known in the literature as the Monkhorst-Pack
  grid, for the Brillouin zone integration.  It is just the good old Gauss quadrature method
  with sine/cosine functions chosen for periodic functions.

  \textit{Default value:}  The $4\times4\times4$ sampling with $1/2$ shift.
  \begin{equation*}
      \begin{matrix}
        4  &  0  &  0  & \quad  0.5 \\
        0  &  4  &  0  & \quad  0.5 \\
        0  &  0  &  4  & \quad  0.5 \\
      \end{matrix}
  \end{equation*}

  Specified as an integer $3 \times 3$ matrix and a real vector.  At present only the diagonal
  elements of the matrix are relevant, but this format will alow an extension of
  the method in the future and better compatibility with {\sc siesta}.

  It has three lines, each with three integers and a real.
  \begin{equation*}
      \begin{matrix}
        m_{11}  &   m_{12}  &   m_{13}  & \quad  d_1 \\
        m_{21}  &   m_{22}  &   m_{23}  & \quad  d_2 \\
        m_{31}  &   m_{32}  &   m_{33}  & \quad  d_3 \\
      \end{matrix}
  \end{equation*}
  In the direction $j$ the Brillouin zone will be divided in $m_{jj}$ sections and
  a point with a shift of $d_j$ will be selected.

  To use only the $\Gamma$ point (molecule in a supercell) use
  \begin{equation*}
      \begin{matrix}
        1  &  0  &  0  & \quad  0.0 \\
        0  &  1  &  0  & \quad  0.0 \\
        0  &  0  &  1  & \quad  0.0 \\
      \end{matrix}
  \end{equation*}

  It is usual to have $m_{jj}$ an even number and $d_j = 0.5$.  For hexagonal crystals
  and the conventional axis, it is also usual/convenient to have $m_{11}=m_{22}$ a
  multiple of three.  These are recipes that minimize the number of irreducible points
  and avoid breaking symmetry.

  If the diagonal elements are non-positive it will default to $m_{ii}=1$,
  printing a warning.  It also warns the presence of unused non-zero off-diagonal values.

%\begin{fdfentry}{XC!Authors}[string]<PZ>
%  \newcommand\xcidx[1]{\index{exchange-correlation!#1}}
\item{\bf  XC.Authors} (string)

  Particular parametrization of the exchange-correlation
  functional.

  \textit{Default value:}  CA

  \begin{itemize}

    \item{CA}

    Local density approximation (LDA). Quantum Monte Carlo
    calculation of the homogeneous
    electron gas by D. M. Ceperley and B. J. Alder,
    Phys. Rev. Lett. \textbf{45},566 (1980), as parametrized by
    J. P. Perdew and A. Zunger, Phys. Rev B \textbf{23}, 5075 (1981)

   \item{PBE}

    GGA of J. P. Perdew, K. Burke and M. Ernzerhof,
    Phys. Rev. Lett. \textbf{77}, 3865 (1996)

   \item{TBL}

   Meta-GGA of Tran and Blaha.
   F. Tran and P. Blaha, Phys. Rev. Lett. 102, 226401 (2009)

  \end{itemize}

\item{\bf Xc.TBL.C} (real)

  Sets Tran-Blaha constant if a positive value.  If it is negative,
  the original constant is used.  A value around 1.09 is usualy a good choice.
  It can be used to ``fine-tune'' the band gap in simulations
  (second-priciples calculations).

  \textit{Default value:}  1.0

\item{\bf TypeOfScfDiag} (string)

  Indicates how the SCF was performed.

  \begin{itemize}

    \item{PW}

    The full plane-wave basis set.  It is the ``usual'' or traditional choice.

    \item{AO}

    Uses the atomic orbitals included in the pseudopotential file.  Not available in very
    old files.  It corresponds to a Linear Combination of Atomic Orbitals (LCAO) calculation.
    It is very fast, but with limited accuracy.  Use for {\it exploratory} runs
    on complex structures.

    \item{AOJC}

    Uses the atomic orbitals included in the pseudopotential file, but improves the LCAO
    wave-functions with a single Jacobian Correction.
    Slower than the AO option but still quite fast.
    It is very convenient to explore
    the Born-Oppenheimer energy surface in optimization and molecular dynamics.
    However one may want to check the {\it final} results with a full PW calculation.

    \item{AOJCPW}

    Follows an AOJC calculation with a full PW calculation.  Final results are the
    practically the same as the normal PW calculation, but may be faster.
    Check if that is the case for your crystals before using..

  \end{itemize}

\item{\bf DualApproximation} (boolean)

   Use the dual approximation.

  \textit{Default value:}  \texttt{.TRUE.}

   It uses a smaller grid for the calculation of the
   effective potential.  Speeds up the calculations with a compromise on precision.
   For molecular dynamics it is very safe.  For geometry optimization with small
   energy cutoffs may not be accurate enough.

\item{\bf ElectronicTemperature} (temperature)

   A real value followed by unit.  The normal unit is the Kelvin, \texttt{k},
   but will accept other energy units, for example \texttt{meV}

  \textit{Default value:} 0 K

   Occupy orbitals with a Fermi-Dirac distribution with that temperature.

\item{\bf TypeOfPotentialMixing} (string)

  \textit{Default value:}  \texttt{BFGS}

   Type of effective potential mixing used to accelerate convergence.
   BFGS is Broyden-Fletcher-Goldfarb-Shanno method.  It has some modificatitions
   for the maximum and minimum values of mixing.
   BROYD1 is Broyden 1st method.
   In a few cases it may be faster, but in most cases is slower,
   and for difficult cases (supercells) it may not converge.
   It has not been tuned in a long time.

\end{itemize}



\subsection{Minor Self Consistent Field parameters}

Other parameters of self-consistency that have less impact, unless ``very wrong''
choices are made.  Probable safe to leave at default values.

\begin{itemize}

\item{\bf MaxSCFIterations} (integer)

   Maximum number of self-consistent iterations.

  \textit{Default value:}  30

   If very large, computing time can be wasted
   in the cases convergence is not achieved.  If very small the code can terminate without a result, also wasting computer time.

\item{\bf ScfTolerance} (real)

   Convergence parameter (atomic units/Hartree) for self-consistency

  \textit{Default value:}  0.00001

   The SCF iterations are deemed converged if the difference between the input
   and output values of all the components of the Fourier transform of
   the effective potential are smaller than this treshold.

   For exploratory runs decrease this parameter.

\item{\bf MaxDiagIterations} (integer)

   Maximum number of Ritz steps in the iterative diagonalization subroutine.

  \textit{Default value:}  40

  Increase if in the last few SCF iterations you still see warnings about
  noisy diagonalization.  In early SCF iterations, those warnings disappear
  if you increase this parameter, but computing time will also increase
  without changing final results.

\item{\bf DiagTolerance} (real)

   Criteria for iterative diagonalization convergence.

   \textit{Default value:}  0.00001

   The iterative diagonalization is deemed converged if the module of the error vector is
   smaller than that value.
   For exploratory runs you can use the value of 0.001.


\end{itemize}


\subsection{Unused Self Consistent Field parameters}

Over the years only one option was left in the code.


\begin{itemize}


\item{\bf TypeOfPseudopotential}

   Type of pseudopotential that is used.

  \textit{Default value:}  \texttt{PSEUKB}

  Kleinman-Bylander separable pseudos are the only available option.
  In older versions, the non-separable pseudo was available (PSEUDO), as
  well as a gaussian integration separable pseudopotential (PSEUGA).
  Some subroutines and variables still keep the \texttt{ga} characters from those days.

\end{itemize}


\subsection{Molecular dynamics parameters}


\begin{itemize}

\item{\bf MD.TypeOfRun} (string)

   Choice on how the atoms move.  Molecular dynamics or geometry optimization.

  \textit{Default value:}  ONE

   \begin{itemize}

      \item{ONE}

      Just one SCF calculation.  Atoms do not move.

      \item{RSTRT}

      Restart the molecular dynamics from the last atomic configuration.
      Use in case it was interrupted by some external reason (power failure).
      Just keep all the other parameters the same.
      It will reproduce what a non-interrupted calculation would find
      in the case of molecular dynamics.  For optimization the minimization restarts,
      so it will not reproduce a non-interrupted calculation, but probably will converge
      to the same minimum.

      \item{MICRO}

      Microcanonic molecular dynamics.  The total/free energy is conserved, depending
      whether the temperature is zero and the system is an insulator or temperature is finite
      and not too low in the case of a metal.  How low the temperature can go depends on the
      density of integration k-points.

      \item{LANG}

      Langevin molecular dynamics, in contact with a thermostat with temperature
      indicated by {\bf MD.TargetTemperature}.

      \item{VCSLNG}

      Langevin molecular dynamics but with a variational cell shape, that is the lattice vectors
      change with simulation time.

      \item{LBFSYM}

      Minimization of the energy with respect to atomic positions with the LBF algorithm.

      \item{VCSLBF}

      Minimization of the energy with respect to atomic positions and cell shape with the LBF algorithm.

      \item{EPILBF}

      Minimization of the energy with respect to atomic positions and ``vertical'' dimension
      of the cell with the LBF algorithm.  It models an epitaxial situation.
      The initial $\vec a_1$ and $\vec a_2$ of {\bf LatticeVectors} define the
      epitaxial surface.  The cell only contracts and expands in the direction perpendicular
      to those vectors ($\vec b_3$).

   \end{itemize}

   \item{\bf MD.InitialTemperature} (temperature)

   A real value followed by unit.  The normal unit is the Kelvin, \texttt{k},
   but will accept other energy units, for example \texttt{meV}

  \textit{Default value:}  1000 K

   The initial kinetic energy is set according to the temperature .
   If the initial potential energy is high, the temperature of the system may be
   quite higher.  If unsure, first do a thermalization with Langevin,
   otherwise the system may break apart.
   Simulations with {\bf TypeOfScfDiag} as AO or AOJC are great for fast thermalization.

   \item{\bf MD.NumberOfSteps} (integer)

   Number of steps of the molecular dynamics run or maximum number of optimization steps.

   \textit{Default value:}  100

   \item{\bf MD.LengthTimeStep} (time)

   A real value followed by unit.  The usual unit are femtoseconds, \texttt{fs},
   but will accept other time units \texttt{s, ns, ps, autime}.

  \textit{Default value:}  2.418884 \texttt{fs} = 100.0 \texttt{autime}

   Time step for molecular dynamics.  If ut is too small atoms barely move, if too large
   the integrator of the molecular dynamics (Verlet) becomes unstable.  Instability
   appears in the non-conservation of energy.  Few femtoseconds should be OK,
   but remember that light atoms move faster (square root of mass scaling).
   Simulations with {\bf TypeOfScfDiag} as AO or AOJC are great for initial checks.

   \item{\bf MD.TargetTemperature} (temperature)

   A real value followed by unit.  The normal unit is the Kelvin, \texttt{k},
   but will accept other energy units, for example \texttt{meV}

  \textit{Default value:}  300 K

   For molecular dynamics with a Langevin thermostat, it is the temperature of
   the thermal bath.

   \item{\bf MD.FrictionFracInvTimeStep} (real)

   The friction coefficient for simulations with a Langevin thermostat.

  \textit{Default value:}  20.0

   It is set with respect to the time step.
   It indicates how many time steps occur until some
   thermalization is achieved.  If small the thermalization is of bad quality,
   if large it will take a long time to thermalize.

   \item{\bf MD.TargetPressure} (pressure)

   A physical value, requiring a real value followed by unit.
   Accepted units are: \texttt{GPa, Pa, atm, Mbar, bar, MPa}.

  \textit{Default value:}  0  \texttt{GPa}

   In a molecular dynamics with variational cell shape, or cell optimiztion,
   it is the applied pressure.

   \item{\bf MD.TargetStress} (block)

   Real $3 \times 3$ matrix of an applied stress in addition to the applied pressure.
   The units are in GPa only.

  \textit{Default value:}  0

   The matrix should be symmetric, and is in lattice coordinates, so some effort
   is needed to understand the orientation of the crystal.

   \item{\bf MD.CellMass} (real)

   Fictitious cell mass for variational cell shape molecular dynamics.  It is in units
   of electron mass.

  \textit{Default value:}  10.0

   Should be fairly larger than one for the cell dynamics to be slow, but not so
   large that the cell dynamics is too slow.
   Simulations with {\bf TypeOfScfDiag} as AO or AOJC are great for initial checks.

   \item{\bf MD.Seed} (integer)

   Seed for the pseudo-random generator for the thermostat and iniial velocity.

   \textit{Default value:} 87697

   Same seed will give the same trajectory.  Different seeds will give
   different trajectories allowing trivial parallelization of simulations.

   \item{\bf MD.UseKeatingCorrections} (boolean)

   \textit{Default value:}  \texttt{.false.}

   For some tetrahedral semiconductors, it provides a correction to LDA
   that reproduces experimental bond lengths.  Do not use with heavy elements (Sn).
   If the structure is not tetrahedral the code stops.

   \item{\bf MD.UseFixedkplusG} (boolean)

  \textit{Default value:} \texttt{.false.}

   Keep a fixed $\vec k + \vec G$ expansion in a variational cell shape
   simulation to avoid the noise of the basis set changing during the simulation.
   Should only be used near equilibrium.  abinit has a neater way of dealing with this noise,
   the code is prepared to use that trick, but it has not been implemented.
   Use {\bf MD.CG.FixedkplusGTol} to define when it kicks in.

   \item{\bf MD.CG.Tolerance} (force)

   A physical value, requiring a real value followed by unit.
   The normal and internal unit is \texttt{har/bohr},
   but will accept \texttt{N, eV/ang, Ry/bohr}.

  \textit{Default value:}  0.0001 \texttt{har/bohr}

   A geometry optimization run stops when the components of the forces on all atoms are smaller
   than this value.

   \item{\bf MD.CG.StepMax} (length)

   A physical value, requiring a real value followed by unit.
   Accepted units are: \texttt{bohr, ang, nm, m}.

  \textit{Default value:}  0.01 \texttt{bohr}

   Maximum displacement of atoms to avoid instabilities in the optimization.

   \item{\bf MD.CG.FixedkplusGTol} (force)

   A physical value, requiring a real value followed by unit.
   The normal and internal unit is \texttt{har/bohr},
   but will accept \texttt{N, eV/ang, Ry/bohr}.

  \textit{Default value:}  0.01 \texttt{har/bohr}

   The fixed $\vec k + \vec G$ is used {\it after} the components of the forces (har/bohr)
   on all atoms are smaller than this value.

\end{itemize}


\subsection{Symmetry}

Treatment of symmetry during a molecular dynamics or optimization run.

\begin{itemize}

\item{\bf UseSymmetry} (boolean)

  \textit{Default value:} \texttt{.TRUE.} for optimization and single
  runs, \texttt{.FALSE.} for molecular dynamics runs.

   Try to maintain the initial symmetry of the system.  It is close to 100\% fiable
   (as close as real number logic allows).

\item{\bf SymmTolerance} (real)

  Tolerance for symmetry identification.

  \textit{Default value:} 0.00001

   If after a symmetry operation the difference between the positions (in lattice coordinates)
   is smaller than this value, the atoms are considered superposed.

\end{itemize}


\subsection{Pseudopotential files}

Where to find the pseudopotential files.  By default they should be on the directory
where the code is run.
The default name of the file is \texttt{Xy{\_}POTKB{\_}F.DAT} where
\texttt{Xy} is the one or two character chemical symbol.

\begin{itemize}

\item{\bf PathToPseudos} (string)

  Directory where the pseudopotentials are found, either full specification,
  or starting from the running directory

  \textit{Default value:}  null string, indicating running directory.


\item{\bf PseudoSuffix} (string)

  Suffix to add to to the chemical symbol to obtain the file name.

  \textit{Default value:}  \texttt{{\_}POTKB{\_}F.DAT}

  It the PathToPseudos is \texttt{tmp} and the PseudoSuffix is \texttt{{\_}POT.DAT}
  the file used for Si would be \texttt{tmp/Si{\_}POT.DAT}.

\end{itemize}


\subsection{Saving wave-functions to disk}

The calculations are more efficiently done keeping the wave-functions in main memory (RAM).
However if the main memory is too small for the problem, either cache trashing occurs,
impacting severely the computing efficiency, or the program aborts.
Saving wave-functions to disk may allow to run the program with just a small penalty,
if a modern disk (SSD, M.2 preferentially) is used.
Useless if the number of k-points is just 1.  Works better for large number of k-points.
Before terminating the calculation the file is deleted.


\begin{itemize}

\item{\bf TapeToSavePsi} (integer)

  Tape (I used them!) number for saving the wave-functions.  If it is smaller than 10,
  wave-functions will be kept in main memory.

  \textit{Default value:}  \texttt{0}

\item{\bf PathToSavePsi} (string)

  Path to save the wave-functions.

  \textit{Default value:}  (empty)

  By default the \texttt{tmp{\_}psi.dat} will be written in the current directory.
  If PathToSavePsi is \texttt{tmp} then the wave-functions will be saved to
  \texttt{tmp/tmp{\_}psi.dat}.


\end{itemize}


\subsection{Unfolding}

If the structure is a supercell of some lattice it may be useful to plot the band structure
in the ``unfolded'' Brillouin zone of the parent structure.
The unfolding procedure was developed within an industrial collaboration (rede project),
it was used in semiconductors with a parent fcc lattice,
so unfolding was only extensively tested in that lattice.

\begin{itemize}

\item{\bf Rede.Superlattice} (block)

   The block has three lines with three integers.  Each line indicates how each of
   the supercell lattice vectors relate to the lattice vectors of the parent structure.

   Taking as an example the supercell described in
   \texttt{Structures/Supercells/cpw\_Si6Ge6\_Si(001).in}, which has 6 Si atomic layers
   and 6 germanium atomic layers on a (001) surface, the block is
   \begin{verbatim}
      %block Rede.Superlattice
                -1     1     0
                 0     0     1
                 3     3    -3
      %endblock Rede.Superlattice
   \end{verbatim}
   the underlying parent fcc structure has lattice vectors $\vec A_1=(0,a/2,a/2)$,
   $\vec A_2=(a/2,0,a/2)$ and $\vec A_3=(a/2,a/2,0)$.  The supercell vectors
   are $\vec a_1=-\vec A_1+\vec A_2 = (-a/2,a/2,0)$,
   $\vec a_2=\vec A_3=(a/2,a/2,0)$ and
   $\vec a_3=3\vec A_1+3\vec A_2-3\vec A_3 = (0,0,3a)$.
   Notice that the in-plane vectors have been rotated by 45$^\circ$,
   their length is $\sqrt{2} a$ as the lattice constant
   is that of Si, it is consistent with epitaxy on Si $(001)$.
   The third vector has a length slightly larger than $3a$ because
   it was allowed to relax in the presence of the larger Ge atoms.

\item{Rede.NumberOfLatticePlanes}

   In a supercell grown on a surface indicates the number of lattice planes.
   It is used to help in plots to determine band alignments.
   In the above example you have six lattice planes (but twelve atomic planes).
   It must be present for unfolding to work.  That is an ``hack'' to deal
   with back compatibility of some files, but do not influence the
   unfolding.  This ``requirement'' should be removed in future releases.

\end{itemize}

\section{Default output}

The code writes to the default output, which can be redirected to a file.
Depending on the choice of {\bf PrintingLevel} the output will be different.
We will use as a first example  the file {\tt Validation/Output/pw\_ref\_01.out}.

\subsection{Initial information}

On the top of the file we have the information about the version of code, and the date
and time it was run.
\begin{verbatim}
     density-functional pseudopotential plane-wave program version 4.99
     run on the 19-Mar-21 at 13:30:54
\end{verbatim}
notice that care was taken to code back the year 2000 bug in the 1990s.

Continuing with that example we find next some information about how many chemical species,
and maximum number of atoms per chemical species the code is using.
\begin{verbatim}
   The values set by size_mxdtyp_mxdatm_esdf are:

   The value of mxdtyp is:     2
   The value of mxdatm is:        8

   input read from cpw.in
\end{verbatim}
notice that the information about which subroutine printed the
information is included in this case.

Next information are the lattice vectors,
\begin{verbatim}
   Primitive Translation Vectors
                        in a.u.                            in lattice units
  a1=  0.125697E+02  0.000000E+00  0.000000E+00         1.225    0.000    0.000
  a2= -0.418990E+01  0.118508E+02  0.000000E+00        -0.408    1.155    0.000
  a3=  0.000000E+00  0.000000E+00  0.725711E+01         0.000    0.000    0.707
\end{verbatim}
preceded by a warning that they were changed,
\begin{verbatim}
   WARNING:   avec(  3,  3)
   changed from       0.707106780000 to       0.707106781187
\end{verbatim}
informing that the code recognized a value very close to $\sqrt{2}/2$ and
changed it.

The line
\begin{verbatim}
      Single geometry calculation
\end{verbatim}
informs that {\bf MD.TypeOfRun} was {ONE}.

To be reproducible the information needed to generate the pseudopotentials
is printed
\begin{verbatim}
    potentials :
    ------------

    Si  ca  nrl  nc
    19-SEP-18 atom 5.804  Improved Troullier - Martinskb-loc= 2
    3s(  2.00)  rc= 1.993p(  2.00)  rc= 1.993d(  0.00)  rc= 1.99
    nql=4000 delql=   0.0100
\end{verbatim}
see the user guide of the pseudopotential generation code for details.

Next the other chosen or default parameters are printed
\begin{verbatim}
    Local Density Approximation (LDA) using
 Ceperley and Alder as parametrized by Perdew and Zunger

 dual aproximation is used

    The energy cutoff for wave-function kinetic energy is      10.000 Hartree

    SCF is converged when the difference in potentials is less then     0.00001000
    Iterative diagonalizationis converged when the error in |h psi - e psi| is less then     0.00001000

    The temperature for electron Fermi distribution is       0.000Kelvin

     Plane-wave basis calculation
\end{verbatim}
see previous section for the keywords associated.

\subsection{Crystal data}

In the output the initial crystal data is printed,
\begin{verbatim}
       CRYSTAL DATA


      1081.02690815      volume

   real-space metric

        157.99703193     -52.66567731       0.00000000      metric  g11,g12,g13
                         157.99703193       0.00000000      metric      g22,g23
                                           52.66567731      metric          g33

     12.56968703      12.56968703       7.25711219   length 1,2,3  (a.u.)

    109.47122063      90.00000000      90.00000000   angle 12,13,23 (degrees)

        position (lattice coord.)         position (cartesian coord. a.u.)     no. type

      0.50000    0.50000    0.00000     0.72571E+01  0.00000E+00  0.00000E+00   1   Si    position
     -0.12500    0.62500    0.00000     0.36286E+01  0.76973E+01  0.00000E+00   2   Si    position
      0.00000    0.00000    0.00000     0.00000E+00  0.00000E+00  0.00000E+00   3   Si    position
      0.37500    0.12500    0.00000     0.36286E+01 -0.25658E+01  0.00000E+00   4   Si    position
      0.00000    0.50000    0.50000     0.36286E+01  0.51316E+01  0.36286E+01   5   Si    position
      0.50000    0.00000    0.50000     0.36286E+01 -0.51316E+01  0.36286E+01   6   Ge    position
      0.37500    0.62500    0.50000     0.72571E+01  0.25658E+01  0.36286E+01   7   Ge    position
     -0.12500    0.12500    0.50000     0.00000E+00  0.25658E+01  0.36286E+01   8   Ge    position
\end{verbatim}
in what should be self-explanatory.

Here we encounter a characteristic of the output, in that we have a value {\tt 1081.02690815}
followed by a ``key'' {\tt volume}.  This allows a quick search for that key in the output.
For example

\noindent\texttt{cpw2000-5.x.y/WORK\$ grep volume ../Validation/Output/pw\_ref\_05.out}

will return the volumes encoutered in Langevin a molecular dynamics
with variational cell shape
\begin{verbatim}
       843.03621271      volume
       844.20060174      volume
       847.56015618      volume
       852.95790991      volume
       860.21678359      volume
       869.14551064      volume
       879.54271598      volume
       891.20293979      volume
         ...
      1092.59744351      volume
      1091.08256914      volume
      1089.72581021      volume
      1088.52562153      volume
      1087.47977919      volume
      1086.58556009      volume
      1085.84052430      volume
      1085.24216798      volume
\end{verbatim}
in a form that would be trivial to plot (for example with {\tt gnuplot}).

\subsection{Crystal symmetry and reciprocal space info}

The code recognizes the space group operations, although does not
find its conventional name (we will later describe how to find it),
it lists the symmetry operations, in lattice coordinates,
first the matrix and than the eventual fractional coordinate.
\begin{verbatim}
    rotation matrices and fractional translations in lattice coordinates

    1     1  0  0     0  1  0     0  0  1         0.0000000000     0.0000000000     0.0000000000    symmetry op.
    2     0 -1  0    -1  0  0     0  0 -1        -0.5000000000    -0.5000000000     0.0000000000    symmetry op.
    3     0 -1  0    -1  0  0     0  0  1        -0.5000000000    -0.5000000000     0.0000000000    symmetry op.
    4     1  0  0     0  1  0     0  0 -1         0.0000000000     0.0000000000     0.0000000000    symmetry op.
\end{verbatim}
we have 4 symmetry operations, the first is the identity and the last is the reflection with respect
to the ``z'' axis.  Remember that the code only uses the metric, and therefore the z-axis,
which in this case is ``obvious'' because angles 13 and 23 are 90 degrees.
In the general case it may be more complicated.

Next the output informs that there are 13117 G-vectors, but only 1868 are unrelated by spatial or temporal
symmetry, and some information about the maximum values of $g_i$ encountered in the expansion
$\vec G = \sum_{i=1,3} g_i \vec b_i$ for the potential and charge densities.
\begin{verbatim}
    13117 G-vectors are set up in    1868 stars,  kmax =   17  17  10
\end{verbatim}
It also informs that of the $4 \times 4 \times 4$ integration grid in
reciprocal space, there are 12 unrelated by symmetry, and that it will
calculate 24 orbitals
\begin{verbatim}
    12 k-points generated by program from parameters :
    n =    4    4    4      s =  0.50  0.50  0.50      nb =   24


  Computing time for starting (s):           0.01
\end{verbatim}

\subsection{The SCF cycle}

The first information about the SCF cycle is the FFT grid size and the maximum and
minimum values of the local self-consistent potential.
\begin{verbatim}
      in fft for local potential n =   24   24   16

  max and min of potential (Hartree)    0.1915   -4.8485    0.0000
\end{verbatim}
In the first iterations one may encounter warnings of the type
\begin{verbatim}
   WARNING       WARNING:   After h_kb_dia
   The estimated error in energy has an accuracy
  of      7.3 digits
\end{verbatim}
First, 7 digit accuracy is already quite good.  To be fast the code does not
try to be very accurate in the early SCF iterations, saving computing time.
As long as the warnings do not persist into the final iteration, these
warnings can be ignored, they are here because they can be helpful if
things go wrong...

The information for iteration 8 is
\begin{verbatim}
  in fft for local potential n =   24   24   16

  max and min of potential (Hartree)    0.1819   -4.7630    0.0000

  the fermi level is at     5.9538 [eV]

 total energy =   -31.8310932988
  computing time for iteration     8          1.890
\end{verbatim}
again {\tt grep} may be used to find out what is happening along the SCF iterations.

\subsection{Potential, energies and forces}

Once self-consistency is achieved, the information about the reciprocal space expansion of
the density and potential are printed,

\begin{verbatim}
   iteration number  9


    i   k-prot         Ek       den     V(out)    V(in)     delta V   Vionic

    1   0  0  0     0.00000  32.00000  -0.33659
    2  -1  0  0     0.14055   0.01194   0.00052   0.00051   0.00001  -0.01183
                              0.00024   0.00001   0.00001  -0.00000  -0.00000
    3  -1  1  0     0.18740   0.01175   0.00039   0.00039   0.00000  -0.01138
                              0.02455   0.00082   0.00081   0.00001  -0.02276
    4  -1 -1  0     0.37480   0.00032  -0.00005  -0.00005  -0.00000   0.00971
                              0.00000   0.00000   0.00000  -0.00000  -0.00000
    5   0  0  1     0.37480   0.00298   0.00018   0.00018   0.00000  -0.02912
                              0.00000   0.00000  -0.00000   0.00000   0.00000
    6  -2  1  0     0.51535   0.01475   0.00007   0.00007   0.00000   0.00858
                              0.00565   0.00006   0.00005   0.00000   0.00000
    7   0 -1 -1     0.51535  -0.01402  -0.00007  -0.00007  -0.00000  -0.00858
                             -0.00344  -0.00003  -0.00003  -0.00000   0.00000
    8  -2  0  0     0.56220   4.98977   0.03271   0.03271  -0.00000  -0.16014
                             -5.03324  -0.03301  -0.03301  -0.00000   0.15190
    9  -1  1  1     0.56220   4.98124   0.03270   0.03270  -0.00000  -0.16014
                              5.02352   0.03299   0.03299   0.00000  -0.15190
\end{verbatim}
after the index, one has the expansion of the G-vector, the kinetic energy associated
(half the length squared) the Fourier coefficient of the density,
the Fourier coefficients of the input and output XC potential, followed by the difference {\tt delta V}
and the coefficients of the ionic pseudopotential.
The self-consistency is done by potential mixing (many codes do density mixing) and the
self-consistency condition is on  {\tt delta V}.

Next we have the details of the total energy
\begin{verbatim}
           Iteration number  9    Energies (Ha)          Changes
           ---------------------

           Alpha Term        =         0.917257
           Kinetic  Energy   =        11.949083         0.000022
           Local PP Energy   =        -8.515343        -0.000104
           Nonlocal Energy   =         5.722961         0.000071
           -------------------------------------
           Harris-Foulkes    =       -31.831093        -0.000000
           -------------------------------------
           Eigenvalue Sum    =         1.020901        -0.000013
           HXC  Correction   =         8.135799         0.000001
           Hartree  Energy   =         2.166313         0.000017
           XC       Energy   =        -9.562418        -0.000005
           Ewald    Energy   =       -33.591688
           -------------------------------------
           Total    Energy   =       -31.831093         0.000000
\end{verbatim}
The last column has the change with respect to the previous iteration,
it is not zero in the contributions, but has six decimals in the variational
total energy, meaning nice convergence.
it should be noticed that the Harris-Foulkes energy is identical (within the
six decimals) with the total energy, another sign of good convergence.

Th total energy is the sum of the five last contributions,
the Harris-Foulkes energy is the sum of the four first contributions
plus the Ewald energy.

The stress tensor is given in both lattice and cartesian coordinates
\begin{verbatim}
        Contravariant stress tensor  (a.u.)               Cartesian stress (GPa)
         0.000642    0.000195    0.000000         0.239973E+01    0.929093E-12    0.267779E-13   stress 1               Total
         0.000195    0.000642    0.000000         0.914063E-12    0.256751E+01   -0.116381E-13   stress 2               Total
         0.000000    0.000000    0.000836         0.254685E-13   -0.104815E-13    0.119894E+01   stress 3               Total

          0.00006986          2.05539463      pressure (au and GPa)                Total
\end{verbatim}
its trace (divided by three) is the pressure.

Final information is the force on the atoms,
\begin{verbatim}
          Force (Lattice coord.)            Force (Cartesian coord. a.u)       no. type

      0.00053   -0.00062    0.00000    -0.69115E-03 -0.11838E-01  0.74502E-17   1   Si    force                Total
     -0.00001    0.00001   -0.00000    -0.80366E-16  0.15823E-03 -0.33131E-15   2   Si    force                Total
      0.00062   -0.00053   -0.00000     0.69115E-03 -0.11838E-01 -0.30332E-15   3   Si    force                Total
     -0.00058    0.00058    0.00000    -0.57519E-15  0.11843E-01  0.15964E-15   4   Si    force                Total
     -0.00053    0.00053    0.00000    -0.14862E-15  0.10863E-01  0.88501E-16   5   Si    force                Total
     -0.00063    0.00063   -0.00000    -0.51521E-15  0.12831E-01 -0.18928E-15   6   Ge    force                Total
     -0.00032   -0.00091   -0.00000    -0.89184E-02 -0.60653E-02 -0.27596E-15   7   Ge    force                Total
      0.00091    0.00032   -0.00000     0.89184E-02 -0.60653E-02 -0.51362E-16   8   Ge    force                Total
\end{verbatim}
indicating that we are close but not at equilibrium.

Computing time is given at the very end
\begin{verbatim}
  Total computing time (s):        47.74    Elapsed time (s):        12.01
\end{verbatim}
with the ratio between computing and elapsed time indicating that the system had four cores.


\section{Other files from cpw.exe}

By design the code tries to write the minimum number of files.  With grep one should be able to
produce relevant information.

For example

\noindent\texttt{cpw2000-5.x.y/WORK\$ grep "4   Si    position" ../Validation/Output/pw\_ref\_04.out}

will produce the trajectory of the 4th atom in the simulation cell
\begin{verbatim}
      0.37456    0.12804    0.00365     0.36474E+01 -0.25300E+01  0.26210E-01   4   Si    position
      0.37435    0.13059    0.01401     0.36644E+01 -0.25017E+01  0.10068E+00   4   Si    position
      0.37460    0.13249    0.02300     0.36800E+01 -0.24848E+01  0.16520E+00   4   Si    position
      0.37504    0.13449    0.02959     0.36977E+01 -0.24688E+01  0.21256E+00   4   Si    position
      0.37525    0.13653    0.03398     0.37141E+01 -0.24500E+01  0.24414E+00   4   Si    position
      0.37561    0.13857    0.03628     0.37314E+01 -0.24328E+01  0.26066E+00   4   Si    position
      ....
      0.37419    0.12132   -0.02379     0.35960E+01 -0.25952E+01 -0.17094E+00   4   Si    position
      0.37595    0.11977   -0.02458     0.35975E+01 -0.26292E+01 -0.17655E+00   4   Si    position
      0.37730    0.11958   -0.02441     0.36060E+01 -0.26451E+01 -0.17536E+00   4   Si    position
      0.37842    0.11900   -0.02299     0.36098E+01 -0.26624E+01 -0.16519E+00   4   Si    position
      0.37909    0.11856   -0.02009     0.36116E+01 -0.26738E+01 -0.14429E+00   4   Si    position
      0.37946    0.11954   -0.01604     0.36213E+01 -0.26676E+01 -0.11524E+00   4   Si    position
\end{verbatim}
allowing the user to plot it.

If the atoms move a {\tt cpw.out} file will be written.  With eventual modifications it can be used
to continue the calculations with the final geometry.  Notice that the default choices are different.

If the atoms move a {\tt RESTART.DAT} file will be present.  It may be used to continue the
simulation for more steps, or restart if the calculation was interrupted by a power failure.

The most important file will be {\tt PW\_RHO\_V.DAT}.  It is a binary file, so it cannot be modified by accident
but that may not allow portability between different operating systems.
It contains the information about the crystal structure, charge density and potential.


\section{Post processing}

Once the self-consistent potential for some structure has been saved to the {\tt PW\_RHO\_V.DAT} file,
one can proceed to calculate many properties witn a post-processing code,

\noindent\texttt{cpw2000-5.x.y/WORK\$  ../cpw\_post\_process.exe}

it is mainly an interactive code, asking questions and proceeding according to the answers.
All the answers are printed back to a file \texttt{replay\_post.dat}.
This way the user will have a record of the answers, and more important,
may rerun the post processing, with eventually slight changes.

\noindent\texttt{cpw2000-5.x.y/WORK\$  cp replay\_post.dat replay.dat}

and then, with optional minor editing of the file,

\noindent\texttt{cpw2000-5.x.y/WORK\$  ../cpw\_post\_process.exe < replay.dat}

The post processing code is in fact several independent subroutines (sub-programs) called from the
same top level program.

That program loops at most 100 times (just edit if you want more!)
asking what the user wants and reads an integer entered by the user.

If the integer is zero, the program stops.  At lower levels of the program a choice of
zero would return the execution to the calling level.

If the integer is not in the allowed range, the question is repeated.
In lower levels of the program an out of range choice gives the user a second chance.
If the user gives again a wrong choice the code returns to the calling level.

Looking at the examples


\subsection{Band structures, density of states, optical response, quantum geometric,$\ldots$}

This is the part of the post-processing that has the most options.
The code gives a summary of the main parameters of the SCF calculation (geometry, pseudopotentials, etc,$\ldots$)
and then informs the user what cutoff was used for the plane-wave expansion
in the SCF calcualtion and asks the user what cutoff (s)he wants to use.
Choosing the same cutoff is a safe option.  In this case only Hartree units are allowed.
The code then asks if it can use the dual approximation, again answering ``yes'' is safe.
It then asks if the user wants to modify other parameters.
those are
\begin{itemize}

   \item{Eigenvalue precision}

   You can change the eigenvalue precision of the iterative diagonalization
   if you are not confortable with the default.  You can manage with a low
   precision in most cases.

   \item{Fermi energy}

   The code will read from the file a reasonable guess of the Fermi energy.
   In several cases it will shift the calculated eigenvalues by that value.
   The estimate is not precise, and if you have a better
   estimate you can enter it.

\end{itemize}
In most cases you can answer `no' to the question.

The code then gives a choice of several types of calculations.

\subsubsection{Band structure}

The first option is to plot a band structure.  For that you need a path in the
Brillouin zone.  That information must in a file called \texttt{BAND\_LINES.DAT}
in the working directory.
In the \texttt{Doc} folder the user can find information
from {\sc Quantum Espresso} and  {\sc elk} that can help write or modify
the \texttt{BAND\_LINES.DAT} file.
In \texttt{Structures/Band\_Lines} the user can find some examples.

An online help to generate a BZ path can be found in \texttt{https://seekpath.materialscloud.io/},
based on Hinuma et. al., Comp. Mat. Sci. {\bf 128}, 140 (2017).
Go to the Crystallography section of this document to find how to get a \texttt{.cif}
or {\sc Quantum Espresso} input file that can be uploaded to that site.
The \texttt{VASP KPOINTS input for LDA/GGA} result can be easily copied/pasted/edited
to the texttt{BAND\_LINES.DAT} format.

An example of a \texttt{BAND\_LINES.DAT} file for an fcc lattice is
\begin{verbatim}
  5       11          0.01
  0.5     0.5     0.5         0.0     0.0     0.0      80       L   Lambda   Gamma
  0.0     0.0     0.0         0.0     0.5     0.5      80       Gamma  Delta  X
  0.0     0.5     0.5         0.25    0.625   0.625    20       X       S     K
  0.25    0.625   0.625       1.0     1.0     1.0      60       K    Sigma   Gamma
  0.5     0.5     0.5         0.25    0.75    0.5      60       L      Q     W
\end{verbatim}
In the first line one must specify the number of lines, 5 in this case, and
how many bands (without spin) one wants to plot.  A third optional value
gives the spacing between calcualted points.

Each subsequent line indicates a pannel in the band structure plot.
The initial and final point are indicated in reciprocal lattice
coordinates.  In the example the conventional choice of
primitive lattice vectors was used.  After the coordinates of the final point
one has the number of points in that pannel, it is {\bf not} used if in the first
line the spacing between points is specified.  Finally the
labels of the first point, line and last point are specified.
The code understands the meaning of the greek letters \texttt{Gamma, Lambda, Delta, Sigma}.

\includegraphics[width=0.8\columnwidth]{band_Ge.pdf}

A band structure for Ge is shown in the figure.  Notice that if the end of one pannel
is the beginning of the next the pannels are joined.  If not a small space is introduced.

The band structure can be calculted by several methods.

\begin{itemize}

   \item{Full plane wave basis diagonalization}

   This is the ``normal'' case,  slower but most precise.
   You will produced files with and without spin-orbit,
   in the {\sc grace} and \texttt{gnuplot} formats.

   The higher quality is the {\sc grace} file, \texttt{band\_so.agr} and \texttt{band.agr}
   for the case with and without spin-orbit.  But you have to have installed the graphical software,
   \texttt{https://plasma-gate.weizmann.ac.il/Grace/}, it is available on most linux distributions.
   However development seems to have stopped.
   It is a good start for publication quality figures.

   Available in all distributions is gnuplot, just type

   \noindent\texttt{cpw2000-5.x.y/WORK\$ gnuplot band\_so.gp}

   and you will have a window with the band structure.

   \item{Diagonalization in a Luttinger-Kohn basis}

   This is very fast but only accurate around the $\Gamma$ point of the Brillouin zone.
   It will produce the similar files as the previous case but with an added \texttt{\_lk}
   in the file name.

   \item{k.p method}

   It is a second order expansion of the hamiltonian in a Luttinger-Kohn basis
   around the $\Gamma$ point.  So it is very similar to the previous method,
   but now he files have a \texttt{\_kp} added to the file name.

   \item{2-k-point Luttinger-Kohn interpolation}

   This is a very good compromise between speed and accuracy, but needs some additional
   information.  It calculates the bands on a few points and then interpolates
   with the GLK method between them.

   The code first asks how many interpolation points you want between calculated points.
   Notice that the values in \texttt{BAND\_LINES.DAT} that control the number of k-points
   is {\it after} interpolation.  Besides the starting and end points on each pannel,
   you need to calculate a few (1 to 3) intermediate points to have a good interpolation,
   so the answer to that quaetion depends on how many final points you are going to have.
   The speedup in computing time is almost the number of intermediate points.
   Accuracy and computing time are again the compromises you have to take.

   The code than asks if you want to accept the defaults of the calculation.
   If unsure accept the defaults.  Otherwise read the code and the relevant article.

   This option only produces the \texttt{grace} figures.  They have an added \texttt{\_lk\_int}
   to the file names.  There is also a \texttt{band\_lk\_ref.agr} file with the reference (fully calculated)
   bands.  It should have visible straight lines if you are using the method efficiently.


\end{itemize}

\subsubsection{Prepare file for density of states}

This option writes files that allow later ploting of the density of states.
It needs a file \texttt{DOS\_MESH.DAT} to be present in the working directory,
and it will write a file \texttt{PW\_DOS.DAT} with the information.

An example of a \texttt{DOS\_MESH.DAT} file is
\begin{verbatim}
    12  8 8 8   0.0 0.0 0.0
\end{verbatim}
it has a single line, the first integer is the desired number of bands, the next three integers are
the mesh in reciprocal space, and the last three are the shift of the mesh.
The section about Brillouin zone sampling explains their meaning.

\subsubsection{Properties for single k-vector}

This option allows to analyze in detail the eigenstates for a single k-vector.
The code asks how many bands the user wants to analyze, followed by
the lattice coordinates of the k-vector.

\subsubsection{Prepare file for dielectric function calculation}

This is similar to the prepare file for density of states.  Follow the instructions
on that section.

\subsubsection{Calculate effective band masses}

This section allows the calculation of effective band masses with three different methods.

\begin{itemize}

   \item{Topological tensor}

   This is the most accurate method.
   It will ask first how many bands (not counting spin) should be calculated.
   Afterwards the code asks which coordinate system the user wants to use,
   and then what are the coordinates of the desired k-point.

   The code will ask if the results should include spin-orbit, and, in the
   affirmative case, which method should be used to diagonalize the Hamiltonian matrix.

   \begin{itemize}

      \item{Full diagonalization}

      Recommended if you can afford a full diagonalize a matrix of the indicated size,
      as it is the most accurate.

      \item{Iterative diagonalization}

      Much faster than full diagonalization for large matrices, but may be subject to noise.

      \item{First order perturbation in spin-orbit}

      Uses perturbation theory instead of matrix diagonalization.

   \end{itemize}

   The code will provide a list of energy levels, and then ask if the user wants
   masses in a particular direction.  If the user answers `yes' it will
   ask what coordinate system the user wants to use and then enter a loop
   where the user can enter directions in the chosen coordinate system,
   and the code will print the effective masses in that direction
   for all bands.

   Once the user indicates that it is done with directions,
   the code will ask if the user wants a figure of
   the masses for one band in all directions. If the user answers `yes'
   the code will ask which band the user wants, and writes a file
   \texttt{band\_}{\it n}[\texttt{\_so}]\texttt{.dat}, where {\it n}
   is the band number, with the relevant information.
   The instructions to produce a figure with Mathematica are printed out.


   \item{k.p method}

   This is a fast method, but in the end it will use a finite difference.
   Read in the next section the dangers of finite differences calculations of masses.
   It will ask first how many bands (not counting spin) should be included
   in the exploratory calculation.  It is not the number of bands
   of the k.p model.
   The code gives an idea of the recommended {\it order of magnitude} of the answer.
   Afterwards the code asks which coordinate system the user wants to use,
   and then what what are the coordinates of the desired k-point.
   It will then give the suggested values of the size of the k.p method.
   Choose one of the suggestions, unless you have a strong reason to use another value.



   \item{Finite differences}

   This is a tricky procedure in the case of band crossings (or near avoided band crossings).
   Use it with the utmost care.

\end{itemize}

\subsubsection{Calculate quantum geometric quantities}

   This section of the code will calculate the Berry curvature, quantum metric,
   orbital magnetic moment and effective mass tensors.
   The analysis will proceed by {\it energy level} not by state.

   The first question asked by the code is how many bands (not counting spin) should be examined.
   It gives as hint the number of occupied states if the system was an insulator.
   Enter a somewhat larger number.
   Afterwards the code asks which coordinate system the user wants to use,
   and then what what are the coordinates of the desired k-point.
   It will then ask if spin-orbit should be taken into account, and, in the
   affirmative case, which method should be used to diagonalize the Hamiltonian matrix.
   See previous section for the meaning of the choices.

   The code will then print the {\it energy levels} with spin polarization for each
   state in that level.  It will then enter a loop asking which level
   should be examined, and which quantity should be printed out.
   The options are
   \begin{itemize}

      \item{All quantities}

      All the options below

      \item{Berry curvature}

      An example of output for a non-degenerate level of hexagonal Te is
      \begin{verbatim}
         Berry curvature tensor for level with energy         5.480


    primitive (reciprocal) lattice coordinates

       0.000    -189.681      -0.000
     189.681       0.000       0.000
       0.000      -0.000       0.000


    Cartesian lattice coordinates

       0.000    -292.927      -0.000                        berry_curv    1    1   18
     292.927       0.000       0.000                        berry_curv    1    2   18
       0.000      -0.000       0.000                        berry_curv    1    3   18



   Trace over coordinates       0.000



    Associated pseudo-vector

            0.000       0.000    -292.927                    berry_curvvector   18

    Module:      292.927

      \end{verbatim}
      The $3\times3$ tensor is given in lattice coordinates and cartesian coordinates.
      The columns and rows correspond to the $x$, $y$ and $z$ directions.
      The meaning of those directions are with respect to the conventional orientation
      indicated at the top of the SCF output.  In this case it is what we ``expect'' from
      an hexagonal crystal.  The tensor is antisymmetric, only $\Omega_{xy}=-\Omega_{yx}$
      are non-zero.  The antisymmetric tensor is converted to a pseudo-vector with only
      the $\Omega_z$ component.  The trace is trivially zero.
      Notice that some quantities are followed by keywords so that they can be found by a
      \texttt{grep} command.

      The output for a degenerate level is more complicated,
      here again an output for hexagonal Te, at another k-point,
      for a double degenerate energy level.
      \begin{verbatim}
   Berry curvature tensor for level with energy         4.464


    primitive (reciprocal) lattice coordinates

       0.000       6.017      -0.000          -0.000      -0.000       0.101
      -6.017       0.000      -0.000          -0.000      -0.000       0.201
       0.000       0.000       0.000          -0.101      -0.201      -0.000

       0.000       0.000       0.101           0.000      -6.017       0.000
       0.000       0.000       0.201           6.017       0.000       0.000
      -0.101      -0.201       0.000          -0.000      -0.000       0.000


    Cartesian lattice coordinates

      -0.000       9.292      -0.000          -0.000      -0.000      -0.000                        berry_curv    1    1    9
      -9.292       0.000      -0.000           0.000      -0.000       0.413                        berry_curv    1    2    9
       0.000       0.000       0.000          -0.000      -0.413      -0.000                        berry_curv    1    3    9

       0.000      -0.000       0.000           0.000      -9.292       0.000                        berry_curv    2    1    9
       0.000       0.000       0.413           9.292       0.000       0.000                        berry_curv    2    2    9
       0.000      -0.413       0.000          -0.000      -0.000       0.000                        berry_curv    2    3    9



    Trace over coordinates

      -0.000          -0.000
       0.000           0.000

    Trace over trace of coordinates      -0.000


    Trace over energy levels

           -0.000      -0.000      -0.000
            0.000       0.000      -0.000
            0.000       0.000       0.000


    Pseudo-vector associated with the trace.

           -0.000       0.000      -0.000                    berry_curvvector    9

    Module:        0.000
      \end{verbatim}
      Now we have a $2 \times 2$ group of $3 times 3$ matrices.
      The rows and columns of the $3 times 3$ matrices are the directions $x$, $y$ and $z$,
      the rows and columns of the $2 \times 2$ grouping are associated
      with each state within the doubly degenerate level.
      As care was taken to choose the two states that span the degenerate energy
      eigenspace, there is a nice apparent structure to this four index tensor.


      \item{Quantum metric}

      The quantum metric is a symmetric tensor.
      An example of output for a non-degenerate level of hexagonal Te is
      \begin{verbatim}
Quantum metric tensor for level with energy         5.480


    primitive (reciprocal) lattice coordinates

     244.291     122.146      -0.000
     122.146     244.291      -0.000
      -0.000      -0.000     163.410


    Cartesian lattice coordinates

     326.719      -0.000      -0.000                        qua_metric    1    1   18
      -0.000     326.719      -0.000                        qua_metric    1    2   18
      -0.000      -0.000     516.225                        qua_metric    1    3   18



   Trace over coordinates    1169.664


   Trace over trace of levels    1169.664


   Eigenvalues:      326.719     326.719     516.225
        0.551E+08   Determinant qua_metric   18
      \end{verbatim}
      The structure of the results is similar to the case of
      Berry curvature, but as the tensor is symmetric the trace is non-zero.
      The code calculates a few of the invariants of the tensor.

      An example for the degenerate case is
      \begin{verbatim}
Quantum metric tensor for level with energy         5.793


    primitive (reciprocal) lattice coordinates

     238.414     119.207      -0.000         -33.231      33.231     -29.478
     119.207     238.414      -0.000          33.231      66.463      -5.677
      -0.000      -0.000      58.036          23.800      -5.677       0.000

     -33.231      33.231      23.800         238.414     119.207      -0.000
      33.231      66.463      -5.677         119.207     238.414      -0.000
     -29.478      -5.677       0.000          -0.000      -0.000      58.036


    Cartesian lattice coordinates

     318.858       0.000       0.000         -88.889       0.000     -63.227                        qua_metric    1    1   13
       0.000     318.858      -0.000          -0.000      88.889     -11.670                        qua_metric    1    2   13
       0.000      -0.000     183.341          63.227     -11.670       0.000                        qua_metric    1    3   13

     -88.889      -0.000      63.227         318.858       0.000      -0.000                        qua_metric    2    1   13
       0.000      88.889     -11.670           0.000     318.858      -0.000                        qua_metric    2    2   13
     -63.227     -11.670       0.000          -0.000      -0.000     183.341                        qua_metric    2    3   13



    Trace over coordinates

     821.058          -0.000
      -0.000         821.058

    Trace over trace of coordinates    1642.116


   Eigenvalues:      821.058     821.058
        0.674E+06   Determinant qua_metric   13


    Trace over energy levels

          637.717       0.000      -0.000
            0.000     637.717      -0.000
           -0.000      -0.000     366.682

   Trace over trace of levels    1642.116


   Eigenvalues:      366.682     637.717     637.717
        0.149E+09   Determinant qua_metric   13
      \end{verbatim}
      Besides the full tensor the code also calculate some traces and determinants.


      \item{Orbital magnetic momentum}

      The orbital magnetic momentum is antisymmetric like the Berry curvature.  See that case for details.

      \item{Effective mass tensor}

      The effective mass tensor is symmetric like the quantum metric.  See that case for details.

      \item{Interband contribution to mass tensor}

      The effective mass tensor has two contributions, the interband contribution is the real part
      of a complex tensor whose imaginary part is the orbital magnetic momentum.
      Here it is printed separately so the full complex tensor is available.

   \end{itemize}


\subsubsection{Reset dual approximation flag}

This option allows to toggle the use of the dual approximation.
Useful if you want to check its consequences.



\subsection{Modified tight binding}

For a fast dense sampling of the Brillouin zone, a modified tight-binding scheme
can be used.  It has many similarities with Wannier interpolation, and in fact
can be used to write the files required to call {\sc wannier90}.

\subsection{Plot of charge densities and potentials}

Plots or prepares the plots of real functions electron density, $\rho(\vec r)$, effective potential,
$v_\text{eff}(\vec r)$ and some potential contributions.

It can generate a three dimensional plot in the \texttt{".xsf} format of {\sc XCrysDen}
compatible with {\sc vesta}, or a two dimensional level plot in a plane chosen by the user.

An interesting capability is to plot along a line these quantities averaged over the
perpendicular plane.  It is very useful to find band alignments.

\subsection{Plot densities of states}

The data for the calculation of the density of states and respective plot
can be generated with different approaches.  This section of
post processing reads the file with the data and generates the plot.

Choices of graphics options are separated from the sometimes very heavy calculations of
the bands throughout the Brillouin zone.

\subsection{Exploration of a k.p model}

The first option may store a k.p model in a file.
This section allows the exploration of that model.

\subsection{Plot dielectric function}

Like the case of density of states, the data for the dielectric function
is first calculated throughout the Brillouin zone,
the choices of graphics options are chosen in this section of the code.

\subsection{Plot of the wave-functions}

It is similar to the plots of the charge density, but now the function
to be plotted is complex, adding an extra choice of real part, imaginary part,
or module, plus obviously the choice of which state to plot.
Both 2D and 3D options are available.

\subsection{Crystalography}

This section allows a reveryough examination of the crystal structure.

\subsection{Interface to Quantum Espresso}

Quantum Espresso has many more developers than the present code.
This section writes a QE input file.  The pseudopotential
code can write the required pseudopotentials,
including Troullier-Martins pseudopotentials with spin-orbit
(a feature that is not available in the QE toolchain).
The consistency between the two codes are at the 4th or 5th decimal
place in the total energy.

\section{Where to find crystal structures}

Many example of \texttt{cpw.in} files can be found in the \texttt{Structures} directory.
Besides (almost) all the elements, there are many files collected over the years.

There is an auxiliary program in the toolbox, \texttt{Tools/gen\_PW.f90}
that can help the user to create a \texttt{cpw.in} file by answering some questions.

As mentioned before, the input file was chosen to be very similar to the
{\sc siesta} input file, so the user adapt it easily.

Another option is to convert any crystal file in the \texttt{".xyz} format
which is sufficiently simple to be adapted.

\section{Toolbox}

In \texttt{Tools} directory the user can find a few useful tools.

Most are sufficiently short that the user can easily read and modify the code.

It also has programs that call one of the options of post-processing.
They were used to develop those options.

\subsection{Equation of state}

From a few values of energy as a function of volume (or lattice constant for cubic systems),
a least square fit is applied to the Murnaghan or Birch-Murnaghan equations of
state.  The output is the equilibrium volume, bulk modulus and pressure derivative
of the Bulk modulus.  If the user has several such data sets,
the code will find the transition pressures between those structures.

\subsection{Remove gnuplot commands from 2D plot files}

The 2D plot files are in the gnuplot format, with plotting directives.  To remove the
directives one can run \texttt{Tools/convert\_2Dplot.f90}

\subsection{Crystalography from cpw.in}

Doing a crystalographic analysis after a self-consistent calculation
may be too late.  With \texttt{Tools/cpwin\_geom.f90}
the analysis may be done from the \texttt{cpw.in} file.
Read the crystallography sub-section of the post-processing section
to see how it works.

\subsection{Coulomb potential}

The Coulomb potential for hydrogen can be generate by \texttt{Tools/h\_pot\_kb.f90}

\subsection{Graphical interface for analysis of band structure}

This is a very useful tool.  From a file with the relevant data
it generates a very powerful GUI.  There is an earlier version \texttt{BandInfoUi}
based on the \texttt{grace} plotting package, which is similar to the most recent
package and therefore there is no need to describe in detail.

The most recent package is the \texttt{QtBandViewer}.  It is a \texttt{python}
code that uses the \texttt{Qt} user interface framework (\texttt{https://www.qt.io}).

To install follow the instructions on \texttt{Tools/QtBandViewer/AAAREADME}.
It is strongly recommended to create a python virtual environment under a working directory,
and copy files there.  By some obscure reasons (the writer of the guide does not
use python) the python files should be copied.  To use the tool you will need access to
a \texttt{.bv} file format, by default named \texttt{BAND\_SO.DAT.bv} or
\texttt{BAND.DAT.bv} with the band energies and the \texttt{BAND\_LINES.DAT} file
used to generate those files.  Those data files can be huge so avoid copying
them.

As an example we will use the files generated from the structure
file \texttt{Structures/Supercells/cpw\_Si6Ge6\_Si(001).in} with
unfolding.  The \texttt{BAND\_LINES.DAT} was copied from
\texttt{Structures/Supercells/band\_lines\_Si6Ge6\_Si(001).dat}


once the code is installed, run it

\noindent\texttt{cpw2000-5.x.y/WORK/QtBV\$ python3 QtBandViewer.py}

\noindent
The initial window is shown in the left of the figure.  Press the \texttt{Load} button,
and a new window shown on the right will display the
unfolded band structure.   Notice how the bottom of the conduction
band appear to be in the $\Gamma$--X direction (parallel to the
surface) and there are flat bands in the epitaxial growth direction
$\Gamma$--Z.

\begin{minipage}[b]{0.35\columnwidth}
   \includegraphics[width=\columnwidth]{qtbv_init.png}
\end{minipage}\hskip 0.03\columnwidth
\begin{minipage}[b]{0.60\columnwidth}
    \includegraphics[width=\columnwidth]{qtbv_band_all.png}
\end{minipage}

If now we deselect Si leaving only the Ge contributions, and select the $s$, $p$ and $d$ angular momenta,
we see that ``dots'' at the bottom of the band are still dark, but that the flat band almost disappears.
This indicates that the bottom of the conduction band has a contribution from the Ge layers
and there is a localized band in the Si layer.

\begin{minipage}[b]{0.35\columnwidth}
   \includegraphics[width=\columnwidth]{qtbv_sp_ge.png}
\end{minipage}\hskip 0.03\columnwidth
\begin{minipage}[b]{0.60\columnwidth}
    \includegraphics[width=\columnwidth]{qtbv_sp_ge_res.png}
\end{minipage}

Reselecting Si and deselecting Ge and keeping the $s$, $p$ and $d$ angular momenta,
we confirm that the flat band is indeed mostly Si, the bottom of the conduction
band has a Si contribution that is weaker than the Ge contribution.
It is also apparent that for the valence band at the unfolded $\Gamma$
there is a band below the top and split-off bands that has a stronger
Si content.  For larger supercell periodicities it would converge to the band
alignment between Si and strained Ge.


\begin{minipage}[b]{0.35\columnwidth}
   \includegraphics[width=\columnwidth]{qtbv_sp_si.png}
\end{minipage}\hskip 0.03\columnwidth
\begin{minipage}[b]{0.60\columnwidth}
    \includegraphics[width=\columnwidth]{qtbv_sp_si_res.png}
\end{minipage}






\end{document}
